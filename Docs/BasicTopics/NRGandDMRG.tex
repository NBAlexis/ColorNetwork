\subsection{\label{NRGandDMRG}NRG and DMRG}

Here we will introduce numerical renormalization group and density matrix renormalization group using an example to solve an equation of a single particle.

We will consider solving such an equation in range $x\in [-1, 1]$.
\begin{equation}
\begin{split}
&\frac{\partial ^2 }{\partial x^2}f(x)=af(x),\\
&f(-1)=f(1)=0,\\
\end{split}
\end{equation}
with 
\begin{equation}
\begin{split}
&\frac{\partial ^2 }{\partial x^2}f(x)\approx \frac{f(x+\Delta x)+f(x-\Delta x)-2f(x)}{\Delta  x ^2},\\
\end{split}
\end{equation}
the discretized version is an eigen-problem
\begin{equation}
\begin{split}
&\frac{1}{\Delta x^2}\begin{pmatrix}
2 & -1 & 0 & 0 & \ldots & 0 & 0 \\
-1 & 2 & -1 & 0 & \ldots & 0 & 0 \\
0 & -1 & 2 & -1 & \ldots & 0 & 0 \\
0 & 0 & -1 & 2 & \ldots & 0 & 0 \\
\ldots & \ldots & \ldots & \ldots & \ldots & \ldots & \ldots \\
0 & 0 & 0 & 0 & \ldots & 2 & -1 \\
0 & 0 & 0 & 0 & \ldots & -1 & 2 \\
\end{pmatrix}f=a f,\\
\end{split}
\end{equation}

It can be written as
\begin{equation}
\begin{split}
&\frac{1}{\Delta x^2}\begin{pmatrix}
H_0 & L_0 & 0 & 0 & \ldots & 0 & 0 \\
L_0 & H_0 & L_0 & 0 & \ldots & 0 & 0 \\
0 & L_0 & H_0 & L_0 & \ldots & 0 & 0 \\
0 & 0 & L_0 & H_0 & \ldots & 0 & 0 \\
\ldots & \ldots & \ldots & \ldots & \ldots & \ldots & \ldots \\
0 & 0 & 0 & 0 & \ldots & H_0 & L_0 \\
0 & 0 & 0 & 0 & \ldots & L_0 & H_0 \\
\end{pmatrix}f\approx 
\frac{1}{\Delta x^2}\begin{pmatrix}
H_0 & 0 & 0 & 0 & \ldots & 0 & 0 \\
0 & H_0 & 0 & 0 & \ldots & 0 & 0 \\
0 & 0 & H_0 & 0 & \ldots & 0 & 0 \\
0 & 0 & 0 & H_0 & \ldots & 0 & 0 \\
\ldots & \ldots & \ldots & \ldots & \ldots & \ldots & \ldots \\
0 & 0 & 0 & 0 & \ldots & H_0 & 0 \\
0 & 0 & 0 & 0 & \ldots & 0 & H_0 \\
\end{pmatrix}
=a f,\\
\end{split}
\end{equation}
with
\begin{equation}
\begin{split}
&H_0=\begin{pmatrix}
2 & -1 & 0 & 0 \\
-1 & 2 & -1 & 0 \\
0 & -1 & 2 & -1 \\
0 & 0 & -1 & 2 \\
\end{pmatrix},\;\;L_0=\begin{pmatrix}
0 & 0 & 0 & -1 \\
0 & 0 & 0 & 0 \\
0 & 0 & 0 & 0 \\
0 & 0 & 0 & 0 \\
\end{pmatrix}
\end{split}
\end{equation}    

The first step is to solve eigen system of $H$, the two lowest eigen values and eigen vectors are
\begin{equation}
\begin{split}
&H_0\begin{pmatrix}
\frac{1}{5+\sqrt{5}} & -\frac{1}{5-\sqrt{5}} \\
\frac{1}{2\sqrt{5}} & -\frac{1}{2\sqrt{5}} \\
\frac{1}{2\sqrt{5}} & \frac{1}{2\sqrt{5}} \\
\frac{1}{5+\sqrt{5}} & \frac{1}{5-\sqrt{5}} \\
\end{pmatrix}=
\left(
\frac{3-\sqrt{5}}{2}\times\begin{array}{c}
\frac{1}{5+\sqrt{5}}\\
\frac{1}{2\sqrt{5}}\\
\frac{1}{2\sqrt{5}}\\
\frac{1}{5+\sqrt{5}}
\end{array}
\frac{5-\sqrt{5}}{2}\times\begin{array}{c}
-\frac{1}{5-\sqrt{5}}\\
-\frac{1}{2\sqrt{5}}\\
\frac{1}{2\sqrt{5}}\\
\frac{1}{5-\sqrt{5}}\\
\end{array}
\right)
\end{split}
\end{equation} 

Then, truncate use
\begin{equation}
\begin{split}
&U=\begin{pmatrix}
\frac{1}{5+\sqrt{5}} & -\frac{1}{5-\sqrt{5}} \\
\frac{1}{2\sqrt{5}} & -\frac{1}{2\sqrt{5}} \\
\frac{1}{2\sqrt{5}} & \frac{1}{2\sqrt{5}} \\
\frac{1}{5+\sqrt{5}} & \frac{1}{5-\sqrt{5}} \\
\end{pmatrix}\\
&H'=U^T H_0 U = \begin{pmatrix}\frac{1}{2}-\frac{1}{\sqrt{5}} & 0 \\ 0 & \frac{1}{2}\end{pmatrix},\;\;
L'=U^T L_0 U = \begin{pmatrix}-\frac{1}{(5+\sqrt{5})^2} & -\frac{1}{20} \\ \frac{1}{20} & \frac{1}{(5-\sqrt{5})^2}\end{pmatrix}.
\end{split}
\end{equation}
Now, we are dealing with a truncated problem
\begin{equation}
\begin{split}
&\frac{1}{\Delta x^2}\begin{pmatrix}
H_1 & L_1 & 0 & 0 & \ldots & 0 & 0 \\
L_1 & H_1 & L_1 & 0 & \ldots & 0 & 0 \\
0 & L_1 & H_1 & L_1 & \ldots & 0 & 0 \\
0 & 0 & L_1 & H_1 & \ldots & 0 & 0 \\
\ldots & \ldots & \ldots & \ldots & \ldots & \ldots & \ldots \\
0 & 0 & 0 & 0 & \ldots & H_1 & L_1 \\
0 & 0 & 0 & 0 & \ldots & L_1 & H_1 \\
\end{pmatrix}f
=a f,\\
\end{split}
\end{equation}
with
\begin{equation}
\begin{split}
&H_1=\begin{pmatrix}
H' & L' \\
L' & H' \\
\end{pmatrix}=\left(
    \begin{array}{cccc}
     \frac{1}{2}-\frac{1}{\sqrt{5}} & 0 & -\frac{1}{\left(\sqrt{5}+5\right)^2} & \frac{1}{20} \\
     0 & \frac{1}{2} & -\frac{1}{20} & \frac{1}{\left(5-\sqrt{5}\right)^2} \\
     -\frac{1}{\left(\sqrt{5}+5\right)^2} & -\frac{1}{20} & \frac{1}{2}-\frac{1}{\sqrt{5}} & 0 \\
     \frac{1}{20} & \frac{1}{\left(5-\sqrt{5}\right)^2} & 0 & \frac{1}{2} \\
    \end{array}
    \right),\\
&L_1=\begin{pmatrix}
0 & L' \\
0 & 0 \\
\end{pmatrix}
\end{split}
\end{equation}  

Again, we solve $H_1f=af$, the lowest two eigenvalues are $0.067$ and $0.026$. As a compare, the lowest two eigenvalues of a $16\times 16$ original matrix are $0.135$ and $0.034$.

